\documentclass{article}
\title{Diferencias entre variables static y final}
\author{David Gabriel Corzo Mcmath}
\date{2019-Nov-14 23:15:56}
%%%%%%%%%%%%%%%%%%%%%%%%%%%%%%%%%%%%%%%%%%%%%%%%%%%%%%%%%%%%%%%%%%%%%%%%%%%%%%%%%%%%%%%%%%%%%%%%%%%%%%%%%%%%%%%%%%%%%%%%%%%%%%%%%%%%%%%%%%%%%%%
\usepackage[margin = 1in]{geometry}
\usepackage{graphicx}
\usepackage{fontenc}
\usepackage{pdfpages}
\usepackage[spanish]{babel}
\usepackage{amsmath}
\usepackage{amsthm}
\usepackage[utf8]{inputenc}
\usepackage{enumitem}
\usepackage{mathtools}
\usepackage{import}
\usepackage{xifthen}
\usepackage{pdfpages}
\usepackage{transparent}
\usepackage{color}
\usepackage{fancyhdr}
\usepackage{lipsum}
\usepackage{sectsty}
\usepackage{titlesec}
\usepackage{calc}
\usepackage{lmodern}
\usepackage{xpatch}
\usepackage{blindtext}
\usepackage{bookmark}
\usepackage{fancyhdr}
\usepackage{xcolor}
\usepackage{tikz}
\usepackage{blindtext}
\usepackage{hyperref}
\usepackage{listing}
\usepackage{spverbatim}
\usepackage{fancyvrb}
\usepackage{fvextra}
\usepackage{amssymb}
\usepackage{pifont}
\usepackage{longtable}
%%%%%%%%%%%%%%%%%%%%%%%%%%%%%%%%%%%%%%%%%%%%%%%%%%%%%%%%%%%%%%%%%%%%%%%%%%%%%%%%%%%%%%%%%%%%%%%%%%%%%%%%%%%%%%%%%%%%%%%%%%%%%%%%%%%%%%%%%%%%%%%
\begin{document}
\maketitle

\section{Static \& Final}
Las palabras ``static'' \& ``final'' en un programa son keywords que pueden ser aplicadas a una clase, variable o método. La gramática para implementar una varibale, claso o método en Java es simplemente poner el keyword ``final'' o ``static'' al iniciar la declaración. Uno de los usos más familiares y recurrentes que tiene en particular las estáticas son que se utilizan para el método ``main'', dicho método como ya se sabe es el punto de entrada al programa; En este ensayo se tratará de explicar cosas como por qué se declara ``main'' como ``static'' y para que nos funciona el keyword ``final''.

\subsection{Definición de Static}
\emph{\textbf{Definición de ``Static":} }







\end{document}
