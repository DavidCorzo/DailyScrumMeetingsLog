\documentclass{article}
\title{Diferencias entre variables static y final}
\author{David Corzo, Anesveth Maatens, Ian Jenatz \& Adriana Mundo}
\date{2019-Nov-14 23:15:56}
%%%%%%%%%%%%%%%%%%%%%%%%%%%%%%%%%%%%%%%%%%%%%%%%%%%%%%%%%%%%%%%%%%%%%%%%%%%%%%%%%%%%%%%%%%%%%%%%%%%%%%%%%%%%%%%%%%%%%%%%%%%%%%%%%%%%%%%%%%%%%%%
\usepackage[margin = 1in]{geometry}
\usepackage{graphicx}
\usepackage{fontenc}
\usepackage{pdfpages}
\usepackage[spanish]{babel}
\usepackage{amsmath}
\usepackage{amsthm}
\usepackage[utf8]{inputenc}
\usepackage{enumitem}
\usepackage{mathtools}
\usepackage{import}
\usepackage{xifthen}
\usepackage{pdfpages}
\usepackage{transparent}
\usepackage{color}
\usepackage{fancyhdr}
\usepackage{lipsum}
\usepackage{sectsty}
\usepackage{titlesec}
\usepackage{calc}
\usepackage{lmodern}
\usepackage{xpatch}
\usepackage{blindtext}
\usepackage{bookmark}
\usepackage{fancyhdr}
\usepackage{xcolor}
\usepackage{tikz}
\usepackage{blindtext}
\usepackage{hyperref}
\usepackage{listing}
\usepackage{spverbatim}
\usepackage{fancyvrb}
\usepackage{fvextra}
\usepackage{amssymb}
\usepackage{pifont}
\usepackage{longtable}
%%%%%%%%%%%%%%%%%%%%%%%%%%%%%%%%%%%%%%%%%%%%%%%%%%%%%%%%%%%%%%%%%%%%%%%%%%%%%%%%%%%%%%%%%%%%%%%%%%%%%%%%%%%%%%%%%%%%%%%%%%%%%%%%%%%%%%%%%%%%%%%
\begin{document}
\maketitle

\section{Static \& Final}
Las palabras ``static'' \& ``final'' en un programa son keywords que pueden ser aplicadas a una clase, variable o método. La gramática para implementar una variable, clase o método en Java es simplemente poner el keyword ``final'' o ``static'' al iniciar la declaración. Uno de los usos más familiares y recurrentes que tiene en particular las estáticas son que se utilizan para el método ``main'', dicho método como ya se sabe es el punto de entrada al programa; En este ensayo se tratará de explicar cosas como por qué se declara ``main'' como ``static'' y para que nos funciona el keyword ``final''.

\section{Definición de Static}
\emph{\textbf{Definición de ``Static":} Static en Java es una keyword que es aplicable a clases, variables, métodos y más; cuando se declara el keyword static antes de un método, variable, etcétera se convierte en global a todos los demás miembros de la clase.} Ahorita cobra cierto valor que el método de entrada al programa (main) por que recordar que se declara como ``\verb|public| \textbf{static} \verb|void main(String[] args) { <<acciones>>; }|''.

\subsubsection{Tipos relevantes de static}
Las variables static tienen ciertas peculiaridades entre ellas, por ejemplo:
    \begin{itemize}
        \item Las variables estáticas tiene la validez de una variable global para cualquier otro miembro de la misma clase.
        \item No se requiere instanciar un objeto de la clase antes de poder instanciar la variable estática.
        \item \emph{\textbf{Ejemplo: }}\verb|nombre_de_la_clase.miembro_static|; // esto puede accesar a esta variable sin instanciarla previamente.  
    \end{itemize}
Los métodos estáticos por otro lado tienen otras características notables que los hacen únicos, y algunas peculiaridades que uno tiene que tener en cuenta si un quiere evadir errores, por ejemplo:
    \begin{itemize}
        \item Los métodos estáticos solo pueden llamar a otros métodos estáticos, sólo pueden accesar información declarada como estática, puede ser accesada de la misma manera que las variables estáticas.
    \end{itemize}

Las clases estáticas por otro lado sólo pueden ser ejecutadas si se instancian primero. \newline 


\section{}





\end{document}
