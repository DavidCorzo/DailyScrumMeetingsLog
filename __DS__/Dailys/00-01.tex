%%%%%%%%%%%%%%%%%%%%%%%%%%%%%%%%%%%%%%%%%%%%%%%%%%%%%%%%%%%%%%%%%%%%%%%%%%%%%%%%%%%%%%%%%%%%%%%%
\section{Lista de stories}
\begin{center}
   \begin{tabular}{ | p{1cm} | p{12cm} | p{1cm} | p{2cm} | }
        \hline
        \# & Descripción & Status & Ponderación \\
        \hline
        000 & C warmup & \checkmark & -- \\ 
        \hline
        001 & Lenguaje interpretado & \checkmark & 1 \\
        \hline
        002 & Vidos sobre los lenguajes interretados y compilados & \checkmark & 1 \\
        \hline
        003 & Implementar: Roll a Ball & \checkmark & 1 \\
        \hline
        004 & Cuadro comparativo OPP - Estructura & \checkmark & 1 \\
        \hline
        005 & Doodle sobre la arquitectura de Java & \checkmark & 1 \\
        \hline
        006 & Queue and Stack & \checkmark & 0.25 \\
        \hline
        007 & Impacto de Java en la actualidad & \checkmark & 1 \\
        \hline
        008 & Cuestionario sobre características de Java & \checkmark & 0.98 \\
        \hline
        009 & Conocer los tipos de datos primitivos de Java & \checkmark & 1 \\
        \hline
        010 & Entender las formas de utilización del sistema I/O de Java & \checkmark & 1 \\
        \hline
        011 & Conocer las estructuras de control de flujo (if, while, for) & \checkmark & 0.90 \\
        \hline
        012 & Ejercicios & \checkmark & 1 \\
        \hline
        013 & Comprender la estructura de métodos en Java & \checkmark & 1 \\
        \hline
        014 & Comprender la diferencia entre la clase y objeto & \checkmark & 1 \\
        \hline
        015 & Constructores y destructores & \checkmark & 1 \\
        \hline
        016 & Comprender las opciones de encapsulamiento de métodos y atributos & \checkmark & 0.97 \\
        \hline
        017 & Aplicación de conceptos en unity & \checkmark &  \\
        \hline
        018 & Ejercicios &  &  \\
        \hline
        019 & Comprender el concepto de sobre carga & \checkmark & p  \\
        \hline
        020 & Comprender la diferencia entre asignar un objeto y clonarlo & \checkmark &   \\
        \hline
        021 & Comprender la implementación de funciones recursivas en Java & \checkmark &   \\
        \hline
        022 & Comprender el uso de variables static y comprender el concepto & &   \\
        \hline
        023 & ¿Cómo funciona el garbage colector? & &   \\
        \hline
        024 & Comprender los tipos de relaciones que existen entre las clases & &   \\
        \hline
        025 & Herencia & &   \\
        \hline
        026 & Array & & \\
        \hline
        027 & Arrays of objects & &   \\
        \hline
        028 & Object of arrays & &   \\ 
        \hline
   \end{tabular}
\end{center}
%%%%%%%%%%%%%%%%%%%%%%%%%%%%%%%%%%%%%%%%%%%%%%%%%%%%%%%%%%%%%%%%%%%%%%%%%%%%%%%%%%%%%%%%%%%%%%%%
\section{Descripción}
\begin{center}
   \begin{longtable}{ | p{18cm} | }
       \hline
        Descripción  \\
       \hline
        001 - Lenguaje interpretado - Presentación \\
        \hline
        002 - Videos sobre los lenguajes interretados y compilados -  Video sobre lenguajes interpretados y compilados \\
        \hline
        003 - Implementar: Roll a Ball - Terminar el tutorial: \url{https://learn.unity.com/project/roll-a-ball-tutorial} \\
        \hline
        004 - Cuadro comparativo OPP - Estructura - Cuadro comparativo OOP, P. Estructurada \\
        \hline
        005 - Doodle sobre la arquitectura de Java - Doodle sobre arquitectura Java \\
        \hline
        006 - Queue and Stack - Implementar un programa en C que muestre el uso de un Stack y de un Queue \\
        \hline
        007 - Impacto de Java en la actualidad - Ensayo sobre el rol de Java en el mundo de la programación  \\
        \hline
        008 - Cuestionario sobre características de Java - Diseñar un cuestionaro de 15 preguntas basados en el capitulo 1 del libro ThinkJava publicado en MiU. \\
        \hline
        009 - Conocer los tipos de datos primitivos de Java - 
        Enumerar los tipos de datos primitivos en java y sus caracteristicas. Entregar un documento escrito \\
        \hline
        010 - Entender las formas de utilización del sistema I/O de Java - Construir un programa que solicite un string al usuario y que lo imprima en mayúsculas. \\
        \hline
        011 - Conocer las estructuras de control de flujo (if, while, for) - Resolver la hoja de ejercicios adjunta a este story. \\
        \hline
        012 - Ejercicios - Resolver la hoja de ejercicios adjunto a este Story. \\
        \hline
        013 - Comprender la estructura de métodos en Java - Entregar un documento explicando la gramática que define a los metodos y procedimientos de Java \\
        \hline
        014 - Comprender la diferencia entre la clase y objeto - Diseñar una clase donde en 10 minutos puedan explicar que es una clase y que es un objeto. Debe mostrar visualmente como se relacionan entre sí. \\
        \hline
        015 - Constructores y destructores - Resuelva el ejercicio descrito en el documento adjunto a este Story \\
        \hline
        016 - Comprender las opciones de encapsulamiento de métodos y atributos - Cada integrante del equipo debe comprender el significado de las opciones de Visibilidad. Evaluación oral. \\
        \hline
        017 - Aplicación de conceptos en unity - Modifique el juego RollABall para que ahora aparezcan cubos de colores en tiempo real y el juego le solicite al jugador colectar cubos de X color. La solicitud del color cambia a traves del tiempo. El juego dura 60 segundos y el objetivo es ver que jugador logra la mayor cantidad de puntos.  \\
        \hline
        018 - Ejercicios - Resolver los ejercicios del archivo adjunto a este Story. \\
        \hline
        019 - Comprender el concepto de sobre carga - Diseñar un programa que ejemplifique un caso de la vida real donde se utilice la sobre carga. \\
        \hline
        020 - Comprender la diferencia entre asignar un objeto y clonarlo -  Elaborar una diagrama que describa el proceso de asignacion y clonación entre objetos.  \\
        \hline
        021 - Comprender la implementación de funciones recursivas en Java - Resolver el ejercicio adjunto al presente Story. \\
        \hline
        022 - Comprender el uso de variables static y comprender el concepto - Elaborar un ensayo que describa el uso de variables Static y Final \\
        \hline
        023 - ¿Cómo funciona el garbage colector? - Diseñe un programa en Unity3D que en combinacion del Profiler le permita analizar el impacto del Garbage Collector sobre el perfomance de una aplicación.
        \begin{itemize}
            \item Debe investigar como funciona el Garbage Collector
            \item Diseñe su programa de manera creativa para "Forzar" la ejecución del garbage collector y con esto tratar de medir su impacto.
        \end{itemize} \\
        \hline
        024 - Comprender los tipos de relaciones que existen entre las clases - Elaborar un video que explique las relaciones posibles entre clases. Utilice como referencia el diagrama de clases de UML. \\
        \hline
        025 - Herencia -  Preparar un tutorial en Unity3D donde expliquen como la herencia se aplica en el diseño y desarrollo de aplicaciones. \newline 
        El entregable es una guía paso a paso del como implementar un programa en Unity3d que aplique el concepto de Herencia y que a la vez muestre los beneficios de aplicar dicho diseño.  \\
        \hline
        026 - Array - Leer el capitulo 12 del libro ThinkJava e implementar los ejercicios adjuntos al presente Story. \\
        \hline
        027 - Arrays of objects - Leer el capitulo 13 del libro ThinkJava e implementar los ejercicios adjuntos al presente Story. \\
        \hline
        028 - Object of arrays - Leer el capitulo 14 del libro ThinkJava e implementar los ejercicios adjuntos al presente Story. \\ 
        \hline
   \end{longtable}
\end{center}

%%%%%%%%%%%%%%%%%%%%%%%%%%%%%%%%%%%%%%%%%%%%%%%%%%%%%%%%%%%%%%%%%%%%%%%%%%%%%%%%%%%%%%%%%%%%%%%%
\section{Introducción}
En esta bitácora se encuentra el registro resumido de los sprints realizados por el grupo en cuestión. 

\begin{enumerate}
    \item Lista de Stories
        \begin{itemize}
            \item Adjunta se encuentra una lista de stories que se han realizado en este semestre.
        \end{itemize}
    
    \item Consideración perliminar con story 000:
        \begin{itemize}
            \item La story 000 se hizo individualmente y consistía de una gama de ejercicios que se debían realizar en el lenguaje de programación C.
        \end{itemize}
    
    \item Se realizaron las reuniones en persona y por medio de WhatsApp según la conveniencia del equipo.
\end{enumerate}


\section{Puestos}
\begin{enumerate}
    \item David Corzo $\rightarrow$ Scrum Master
    \item Anesveth Maatens $\rightarrow$ Product Owner
    \item Ian Jenatz $\rightarrow$ Miembro del equipo de trabajo
    \item Adriana Mundo $\rightarrow$ Miembro del equipo de trabajo
\end{enumerate}
