%%%%%%%%%%%%%%%%%%%%%%%%%%%%%%%%%%%%%%%%%%%%%%%%%%%%%%%%%%%%%%%%%%%%%%%%%%%%%%%%%%%%%%%%%%%%%%%%
\section{Lista de stories}
\begin{center}
   \begin{tabular}{ | p{1cm} | p{13cm} | p{1cm} | p{1.25cm} | }
        \hline
        \# & Descripción & Status & Ponderación \\
        \hline
        000 & C warmup & \checkmark &  \\ 
        001 & Lenguaje interpretado & \checkmark &   \\
        002 & Vidos sobre los lenguajes interretados y compilados & \checkmark &  \\
        003 & Implementar: Roll a Ball & \checkmark &  \\
        004 & Cuadro comparativo OPP - Estructura & \checkmark &  \\
        005 & Doodle sobre la arquitectura de Java & \checkmark &  \\
        006 & Queue and Stack & \checkmark ** &  \\
        007 & Impacto de Java en la actualidad & \checkmark &  \\
        008 & Cuestionario sobre caracteristicas de Java & \checkmark &  \\
        009 & Conocer los tipos de datos primitivos de Java & \checkmark &  \\
        010 & Entender las formas de utilización del sistema I/O de Java & \checkmark &  \\
        011 & Conocer las estructuras de control de flujo (if, while, for) & \checkmark &  \\
        012 & Ejercicios & \checkmark &  \\
        013 & Comprender la estructura de métodos en Java & \checkmark &  \\
        014 & Comprender la diferencia entre la clase y objeto & \checkmark &  \\
        015 & Constructores y destructores & \checkmark &  \\
        016 & Comprender las opciones de encapsulamiento de métodos y atributos & \checkmark &  \\
        017 & Aplicación de conceptos en unity  & &  \\
        018 & Ejercicios &  &  \\
        019 & Comprender el concepto de sobre carga & &   \\
        020 & Comprender la diferencia entre asignar un objeto y clonarlo & &   \\
        021 & Comprender la implementación de funciones recursivas en Java & &   \\
        022 & Comprender el uso de variables static y comprender el concepto & &   \\
        023 & ¿Cómo funciona el garbage colector? & &   \\
        024 & Comprender los tipos de relaciones que existen entre las clases & &   \\
        025 & Herencia & &   \\
        026 & Array & &   \\
        027 & Arrays of objects & &   \\
        028 & Object arrays & &   \\ 
        \hline
   \end{tabular}
\end{center}

%%%%%%%%%%%%%%%%%%%%%%%%%%%%%%%%%%%%%%%%%%%%%%%%%%%%%%%%%%%%%%%%%%%%%%%%%%%%%%%%%%%%%%%%%%%%%%%%
\section{Introducción}
En esta bitácora se encuentra el registro resumido de los sprints realizados por el grupo en cuestión. 

\begin{enumerate}
    \item Lista de Stories
        \begin{itemize}
            \item Adjunta se encuentra una lista de stories que se han realizado en este semestre.
        \end{itemize}
    
    \item Consideración perliminar con story 000:
        \begin{itemize}
            \item La story 000 se hizo individualmente y consistía de una gama de ejercicios que se debían realizar en el lenguaje de programación C.
        \end{itemize}
    
    \item Se realizaron las reuniones en persona y por medio de WhatsApp según la conveniencia del equipo.
\end{enumerate}


\section{Puestos}
\begin{enumerate}
    \item David Corzo $\rightarrow$ Scrum Master
    \item Anesveth Maatens $\rightarrow$ Product Owner
    \item Ian Jenatz $\rightarrow$ Miembro del equipo de trabajo
    \item Adriana Mundo $\rightarrow$ Miembro del equipo de trabajo
\end{enumerate}
